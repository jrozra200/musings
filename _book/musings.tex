\PassOptionsToPackage{unicode=true}{hyperref} % options for packages loaded elsewhere
\PassOptionsToPackage{hyphens}{url}
%
\documentclass[]{book}
\usepackage{lmodern}
\usepackage{amssymb,amsmath}
\usepackage{ifxetex,ifluatex}
\usepackage{fixltx2e} % provides \textsubscript
\ifnum 0\ifxetex 1\fi\ifluatex 1\fi=0 % if pdftex
  \usepackage[T1]{fontenc}
  \usepackage[utf8]{inputenc}
  \usepackage{textcomp} % provides euro and other symbols
\else % if luatex or xelatex
  \usepackage{unicode-math}
  \defaultfontfeatures{Ligatures=TeX,Scale=MatchLowercase}
\fi
% use upquote if available, for straight quotes in verbatim environments
\IfFileExists{upquote.sty}{\usepackage{upquote}}{}
% use microtype if available
\IfFileExists{microtype.sty}{%
\usepackage[]{microtype}
\UseMicrotypeSet[protrusion]{basicmath} % disable protrusion for tt fonts
}{}
\IfFileExists{parskip.sty}{%
\usepackage{parskip}
}{% else
\setlength{\parindent}{0pt}
\setlength{\parskip}{6pt plus 2pt minus 1pt}
}
\usepackage{hyperref}
\hypersetup{
            pdftitle={Musings of a Listless Mind},
            pdfauthor={Jake Rozran},
            pdfborder={0 0 0},
            breaklinks=true}
\urlstyle{same}  % don't use monospace font for urls
\usepackage{longtable,booktabs}
% Fix footnotes in tables (requires footnote package)
\IfFileExists{footnote.sty}{\usepackage{footnote}\makesavenoteenv{longtable}}{}
\usepackage{graphicx,grffile}
\makeatletter
\def\maxwidth{\ifdim\Gin@nat@width>\linewidth\linewidth\else\Gin@nat@width\fi}
\def\maxheight{\ifdim\Gin@nat@height>\textheight\textheight\else\Gin@nat@height\fi}
\makeatother
% Scale images if necessary, so that they will not overflow the page
% margins by default, and it is still possible to overwrite the defaults
% using explicit options in \includegraphics[width, height, ...]{}
\setkeys{Gin}{width=\maxwidth,height=\maxheight,keepaspectratio}
\setlength{\emergencystretch}{3em}  % prevent overfull lines
\providecommand{\tightlist}{%
  \setlength{\itemsep}{0pt}\setlength{\parskip}{0pt}}
\setcounter{secnumdepth}{5}
% Redefines (sub)paragraphs to behave more like sections
\ifx\paragraph\undefined\else
\let\oldparagraph\paragraph
\renewcommand{\paragraph}[1]{\oldparagraph{#1}\mbox{}}
\fi
\ifx\subparagraph\undefined\else
\let\oldsubparagraph\subparagraph
\renewcommand{\subparagraph}[1]{\oldsubparagraph{#1}\mbox{}}
\fi

% set default figure placement to htbp
\makeatletter
\def\fps@figure{htbp}
\makeatother

\usepackage{booktabs}
\usepackage[]{natbib}
\bibliographystyle{apalike}

\title{Musings of a Listless Mind}
\author{Jake Rozran}
\date{2020-03-03}

\begin{document}
\maketitle

{
\setcounter{tocdepth}{1}
\tableofcontents
}
\hypertarget{introduction}{%
\chapter{Introduction}\label{introduction}}

This is not a parenting book, but I will draw on many lessons I've learned in my
short time as a dad. It is a bunch of short stories that have shown me that
there is truly no one right way to do things, just the right way for you. Trust
yourself and you'll find yourself very happy.

\hypertarget{WWFY}{%
\chapter{Whatever Works for You}\label{WWFY}}

When I first found out I would be a father, I read books. Our OB/GYN made us go
to a class. I was nervous, but I wasn't cocky. Every existing parent I knew
wanted to give me advice and I happily listened. I wanted to make sure I knew
what to do with the thing when the thing finally came into the world. I was
excited - and as a perfection seeker, I knew I was going to do this dad thing
right.

The first thing I learned is that it isn't a ``thing,'' it's a ``him,'' and he
rocked my world instantly. It didn't take me long to forget everything I had
read in the books, learned in class, and heard from trusted advisors. That's the
thing that happens when you have a baby boy who decides that 2am is the proper
time to urge mom to tell dad it's time to go to the hospital. By the time he
arrived and all the dust had settled, it was late into the evening. That was
hardly the last sleep deprived day I've had as a father.

I, like so many other fathers, wore the haggard look of an unshaved face and
bloodshot eyes proudly as a badge of honor. All this time, I was tired but I was
``doing it right.'' The second he made an uncomfortable noise, I was changing his
diaper, offering him a bottle, and burping him. No child of mine will be
subjected to ``bad'' parenting.

So\ldots{} I ran around without sleep for a long time (like 14 months or more - I'm
pretty fucking stubborn) before I learned the secret. It is not a bad thing to
take cues from the kid and do the things he likes - so what if the books don't
say to do it that way or if the hottest baby product on the market isn't doing
what it promised to do. If the baby is happy (or asleep), leave him be! Even
better, if I am happy, leave well enough alone and keep on keeping on.

\textbf{DO WHATEVER WORKS FOR YOU} and anything else be damned. Find your style,
happiness, parenting technique, career, whatever and do it that way - if it is
working (and you're not hurting yourself or others), keep on going. Give the
baby a pacifier, or swaddle them, or let them sit in a dirty diaper, or sleep
with ear plugs, or take turns with your partner (if you are so lucky to have
one) having a night off. Smile at the people who look at you funny or are bold
enough to tell you that it isn't the right way - they don't have a fucking clue.

\begin{verbatim}
COME BACK TO THIS SECTION
This applies is more than just parenthood - this is an absolute truth in life. 
If you are a lawyer that wants to become a data scientist, go for it. Make sure 
you remind your lawyer friends what a work-life balance is the next time they 
tell you how hard it is to make partner. 
\end{verbatim}

\bibliography{book.bib,packages.bib}

\end{document}
